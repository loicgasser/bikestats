
% Default to the notebook output style

    


% Inherit from the specified cell style.




    
\documentclass{report}

    
    
    \usepackage{graphicx} % Used to insert images
    \usepackage{adjustbox} % Used to constrain images to a maximum size 
    \usepackage{color} % Allow colors to be defined
    \usepackage{enumerate} % Needed for markdown enumerations to work
    \usepackage{geometry} % Used to adjust the document margins
    \usepackage{amsmath} % Equations
    \usepackage{amssymb} % Equations
    \usepackage{eurosym} % defines \euro
    \usepackage[mathletters]{ucs} % Extended unicode (utf-8) support
    \usepackage[utf8x]{inputenc} % Allow utf-8 characters in the tex document
    \usepackage{fancyvrb} % verbatim replacement that allows latex
    \usepackage{grffile} % extends the file name processing of package graphics 
                         % to support a larger range 
    % The hyperref package gives us a pdf with properly built
    % internal navigation ('pdf bookmarks' for the table of contents,
    % internal cross-reference links, web links for URLs, etc.)
    \usepackage{hyperref}
    \usepackage{longtable} % longtable support required by pandoc >1.10
    \usepackage{booktabs}  % table support for pandoc > 1.12.2
    

    
    
    \definecolor{orange}{cmyk}{0,0.4,0.8,0.2}
    \definecolor{darkorange}{rgb}{.71,0.21,0.01}
    \definecolor{darkgreen}{rgb}{.12,.54,.11}
    \definecolor{myteal}{rgb}{.26, .44, .56}
    \definecolor{gray}{gray}{0.45}
    \definecolor{lightgray}{gray}{.95}
    \definecolor{mediumgray}{gray}{.8}
    \definecolor{inputbackground}{rgb}{.95, .95, .85}
    \definecolor{outputbackground}{rgb}{.95, .95, .95}
    \definecolor{traceback}{rgb}{1, .95, .95}
    % ansi colors
    \definecolor{red}{rgb}{.6,0,0}
    \definecolor{green}{rgb}{0,.65,0}
    \definecolor{brown}{rgb}{0.6,0.6,0}
    \definecolor{blue}{rgb}{0,.145,.698}
    \definecolor{purple}{rgb}{.698,.145,.698}
    \definecolor{cyan}{rgb}{0,.698,.698}
    \definecolor{lightgray}{gray}{0.5}
    
    % bright ansi colors
    \definecolor{darkgray}{gray}{0.25}
    \definecolor{lightred}{rgb}{1.0,0.39,0.28}
    \definecolor{lightgreen}{rgb}{0.48,0.99,0.0}
    \definecolor{lightblue}{rgb}{0.53,0.81,0.92}
    \definecolor{lightpurple}{rgb}{0.87,0.63,0.87}
    \definecolor{lightcyan}{rgb}{0.5,1.0,0.83}
    
    % commands and environments needed by pandoc snippets
    % extracted from the output of `pandoc -s`
    \DefineVerbatimEnvironment{Highlighting}{Verbatim}{commandchars=\\\{\}}
    % Add ',fontsize=\small' for more characters per line
    \newenvironment{Shaded}{}{}
    \newcommand{\KeywordTok}[1]{\textcolor[rgb]{0.00,0.44,0.13}{\textbf{{#1}}}}
    \newcommand{\DataTypeTok}[1]{\textcolor[rgb]{0.56,0.13,0.00}{{#1}}}
    \newcommand{\DecValTok}[1]{\textcolor[rgb]{0.25,0.63,0.44}{{#1}}}
    \newcommand{\BaseNTok}[1]{\textcolor[rgb]{0.25,0.63,0.44}{{#1}}}
    \newcommand{\FloatTok}[1]{\textcolor[rgb]{0.25,0.63,0.44}{{#1}}}
    \newcommand{\CharTok}[1]{\textcolor[rgb]{0.25,0.44,0.63}{{#1}}}
    \newcommand{\StringTok}[1]{\textcolor[rgb]{0.25,0.44,0.63}{{#1}}}
    \newcommand{\CommentTok}[1]{\textcolor[rgb]{0.38,0.63,0.69}{\textit{{#1}}}}
    \newcommand{\OtherTok}[1]{\textcolor[rgb]{0.00,0.44,0.13}{{#1}}}
    \newcommand{\AlertTok}[1]{\textcolor[rgb]{1.00,0.00,0.00}{\textbf{{#1}}}}
    \newcommand{\FunctionTok}[1]{\textcolor[rgb]{0.02,0.16,0.49}{{#1}}}
    \newcommand{\RegionMarkerTok}[1]{{#1}}
    \newcommand{\ErrorTok}[1]{\textcolor[rgb]{1.00,0.00,0.00}{\textbf{{#1}}}}
    \newcommand{\NormalTok}[1]{{#1}}
    
    % Define a nice break command that doesn't care if a line doesn't already
    % exist.
    \def\br{\hspace*{\fill} \\* }
    % Math Jax compatability definitions
    \def\gt{>}
    \def\lt{<}
    % Document parameters
    \title{A preliminary analysis of the Bicincitta data}
		\author{Vishal Sood}
		\date{\today}
		
    
    
    

    % Pygments definitions
    
\makeatletter
\def\PY@reset{\let\PY@it=\relax \let\PY@bf=\relax%
    \let\PY@ul=\relax \let\PY@tc=\relax%
    \let\PY@bc=\relax \let\PY@ff=\relax}
\def\PY@tok#1{\csname PY@tok@#1\endcsname}
\def\PY@toks#1+{\ifx\relax#1\empty\else%
    \PY@tok{#1}\expandafter\PY@toks\fi}
\def\PY@do#1{\PY@bc{\PY@tc{\PY@ul{%
    \PY@it{\PY@bf{\PY@ff{#1}}}}}}}
\def\PY#1#2{\PY@reset\PY@toks#1+\relax+\PY@do{#2}}

\expandafter\def\csname PY@tok@gd\endcsname{\def\PY@tc##1{\textcolor[rgb]{0.63,0.00,0.00}{##1}}}
\expandafter\def\csname PY@tok@gu\endcsname{\let\PY@bf=\textbf\def\PY@tc##1{\textcolor[rgb]{0.50,0.00,0.50}{##1}}}
\expandafter\def\csname PY@tok@gt\endcsname{\def\PY@tc##1{\textcolor[rgb]{0.00,0.27,0.87}{##1}}}
\expandafter\def\csname PY@tok@gs\endcsname{\let\PY@bf=\textbf}
\expandafter\def\csname PY@tok@gr\endcsname{\def\PY@tc##1{\textcolor[rgb]{1.00,0.00,0.00}{##1}}}
\expandafter\def\csname PY@tok@cm\endcsname{\let\PY@it=\textit\def\PY@tc##1{\textcolor[rgb]{0.25,0.50,0.50}{##1}}}
\expandafter\def\csname PY@tok@vg\endcsname{\def\PY@tc##1{\textcolor[rgb]{0.10,0.09,0.49}{##1}}}
\expandafter\def\csname PY@tok@m\endcsname{\def\PY@tc##1{\textcolor[rgb]{0.40,0.40,0.40}{##1}}}
\expandafter\def\csname PY@tok@mh\endcsname{\def\PY@tc##1{\textcolor[rgb]{0.40,0.40,0.40}{##1}}}
\expandafter\def\csname PY@tok@go\endcsname{\def\PY@tc##1{\textcolor[rgb]{0.53,0.53,0.53}{##1}}}
\expandafter\def\csname PY@tok@ge\endcsname{\let\PY@it=\textit}
\expandafter\def\csname PY@tok@vc\endcsname{\def\PY@tc##1{\textcolor[rgb]{0.10,0.09,0.49}{##1}}}
\expandafter\def\csname PY@tok@il\endcsname{\def\PY@tc##1{\textcolor[rgb]{0.40,0.40,0.40}{##1}}}
\expandafter\def\csname PY@tok@cs\endcsname{\let\PY@it=\textit\def\PY@tc##1{\textcolor[rgb]{0.25,0.50,0.50}{##1}}}
\expandafter\def\csname PY@tok@cp\endcsname{\def\PY@tc##1{\textcolor[rgb]{0.74,0.48,0.00}{##1}}}
\expandafter\def\csname PY@tok@gi\endcsname{\def\PY@tc##1{\textcolor[rgb]{0.00,0.63,0.00}{##1}}}
\expandafter\def\csname PY@tok@gh\endcsname{\let\PY@bf=\textbf\def\PY@tc##1{\textcolor[rgb]{0.00,0.00,0.50}{##1}}}
\expandafter\def\csname PY@tok@ni\endcsname{\let\PY@bf=\textbf\def\PY@tc##1{\textcolor[rgb]{0.60,0.60,0.60}{##1}}}
\expandafter\def\csname PY@tok@nl\endcsname{\def\PY@tc##1{\textcolor[rgb]{0.63,0.63,0.00}{##1}}}
\expandafter\def\csname PY@tok@nn\endcsname{\let\PY@bf=\textbf\def\PY@tc##1{\textcolor[rgb]{0.00,0.00,1.00}{##1}}}
\expandafter\def\csname PY@tok@no\endcsname{\def\PY@tc##1{\textcolor[rgb]{0.53,0.00,0.00}{##1}}}
\expandafter\def\csname PY@tok@na\endcsname{\def\PY@tc##1{\textcolor[rgb]{0.49,0.56,0.16}{##1}}}
\expandafter\def\csname PY@tok@nb\endcsname{\def\PY@tc##1{\textcolor[rgb]{0.00,0.50,0.00}{##1}}}
\expandafter\def\csname PY@tok@nc\endcsname{\let\PY@bf=\textbf\def\PY@tc##1{\textcolor[rgb]{0.00,0.00,1.00}{##1}}}
\expandafter\def\csname PY@tok@nd\endcsname{\def\PY@tc##1{\textcolor[rgb]{0.67,0.13,1.00}{##1}}}
\expandafter\def\csname PY@tok@ne\endcsname{\let\PY@bf=\textbf\def\PY@tc##1{\textcolor[rgb]{0.82,0.25,0.23}{##1}}}
\expandafter\def\csname PY@tok@nf\endcsname{\def\PY@tc##1{\textcolor[rgb]{0.00,0.00,1.00}{##1}}}
\expandafter\def\csname PY@tok@si\endcsname{\let\PY@bf=\textbf\def\PY@tc##1{\textcolor[rgb]{0.73,0.40,0.53}{##1}}}
\expandafter\def\csname PY@tok@s2\endcsname{\def\PY@tc##1{\textcolor[rgb]{0.73,0.13,0.13}{##1}}}
\expandafter\def\csname PY@tok@vi\endcsname{\def\PY@tc##1{\textcolor[rgb]{0.10,0.09,0.49}{##1}}}
\expandafter\def\csname PY@tok@nt\endcsname{\let\PY@bf=\textbf\def\PY@tc##1{\textcolor[rgb]{0.00,0.50,0.00}{##1}}}
\expandafter\def\csname PY@tok@nv\endcsname{\def\PY@tc##1{\textcolor[rgb]{0.10,0.09,0.49}{##1}}}
\expandafter\def\csname PY@tok@s1\endcsname{\def\PY@tc##1{\textcolor[rgb]{0.73,0.13,0.13}{##1}}}
\expandafter\def\csname PY@tok@kd\endcsname{\let\PY@bf=\textbf\def\PY@tc##1{\textcolor[rgb]{0.00,0.50,0.00}{##1}}}
\expandafter\def\csname PY@tok@sh\endcsname{\def\PY@tc##1{\textcolor[rgb]{0.73,0.13,0.13}{##1}}}
\expandafter\def\csname PY@tok@sc\endcsname{\def\PY@tc##1{\textcolor[rgb]{0.73,0.13,0.13}{##1}}}
\expandafter\def\csname PY@tok@sx\endcsname{\def\PY@tc##1{\textcolor[rgb]{0.00,0.50,0.00}{##1}}}
\expandafter\def\csname PY@tok@bp\endcsname{\def\PY@tc##1{\textcolor[rgb]{0.00,0.50,0.00}{##1}}}
\expandafter\def\csname PY@tok@c1\endcsname{\let\PY@it=\textit\def\PY@tc##1{\textcolor[rgb]{0.25,0.50,0.50}{##1}}}
\expandafter\def\csname PY@tok@kc\endcsname{\let\PY@bf=\textbf\def\PY@tc##1{\textcolor[rgb]{0.00,0.50,0.00}{##1}}}
\expandafter\def\csname PY@tok@c\endcsname{\let\PY@it=\textit\def\PY@tc##1{\textcolor[rgb]{0.25,0.50,0.50}{##1}}}
\expandafter\def\csname PY@tok@mf\endcsname{\def\PY@tc##1{\textcolor[rgb]{0.40,0.40,0.40}{##1}}}
\expandafter\def\csname PY@tok@err\endcsname{\def\PY@bc##1{\setlength{\fboxsep}{0pt}\fcolorbox[rgb]{1.00,0.00,0.00}{1,1,1}{\strut ##1}}}
\expandafter\def\csname PY@tok@mb\endcsname{\def\PY@tc##1{\textcolor[rgb]{0.40,0.40,0.40}{##1}}}
\expandafter\def\csname PY@tok@ss\endcsname{\def\PY@tc##1{\textcolor[rgb]{0.10,0.09,0.49}{##1}}}
\expandafter\def\csname PY@tok@sr\endcsname{\def\PY@tc##1{\textcolor[rgb]{0.73,0.40,0.53}{##1}}}
\expandafter\def\csname PY@tok@mo\endcsname{\def\PY@tc##1{\textcolor[rgb]{0.40,0.40,0.40}{##1}}}
\expandafter\def\csname PY@tok@kn\endcsname{\let\PY@bf=\textbf\def\PY@tc##1{\textcolor[rgb]{0.00,0.50,0.00}{##1}}}
\expandafter\def\csname PY@tok@mi\endcsname{\def\PY@tc##1{\textcolor[rgb]{0.40,0.40,0.40}{##1}}}
\expandafter\def\csname PY@tok@gp\endcsname{\let\PY@bf=\textbf\def\PY@tc##1{\textcolor[rgb]{0.00,0.00,0.50}{##1}}}
\expandafter\def\csname PY@tok@o\endcsname{\def\PY@tc##1{\textcolor[rgb]{0.40,0.40,0.40}{##1}}}
\expandafter\def\csname PY@tok@kr\endcsname{\let\PY@bf=\textbf\def\PY@tc##1{\textcolor[rgb]{0.00,0.50,0.00}{##1}}}
\expandafter\def\csname PY@tok@s\endcsname{\def\PY@tc##1{\textcolor[rgb]{0.73,0.13,0.13}{##1}}}
\expandafter\def\csname PY@tok@kp\endcsname{\def\PY@tc##1{\textcolor[rgb]{0.00,0.50,0.00}{##1}}}
\expandafter\def\csname PY@tok@w\endcsname{\def\PY@tc##1{\textcolor[rgb]{0.73,0.73,0.73}{##1}}}
\expandafter\def\csname PY@tok@kt\endcsname{\def\PY@tc##1{\textcolor[rgb]{0.69,0.00,0.25}{##1}}}
\expandafter\def\csname PY@tok@ow\endcsname{\let\PY@bf=\textbf\def\PY@tc##1{\textcolor[rgb]{0.67,0.13,1.00}{##1}}}
\expandafter\def\csname PY@tok@sb\endcsname{\def\PY@tc##1{\textcolor[rgb]{0.73,0.13,0.13}{##1}}}
\expandafter\def\csname PY@tok@k\endcsname{\let\PY@bf=\textbf\def\PY@tc##1{\textcolor[rgb]{0.00,0.50,0.00}{##1}}}
\expandafter\def\csname PY@tok@se\endcsname{\let\PY@bf=\textbf\def\PY@tc##1{\textcolor[rgb]{0.73,0.40,0.13}{##1}}}
\expandafter\def\csname PY@tok@sd\endcsname{\let\PY@it=\textit\def\PY@tc##1{\textcolor[rgb]{0.73,0.13,0.13}{##1}}}

\def\PYZbs{\char`\\}
\def\PYZus{\char`\_}
\def\PYZob{\char`\{}
\def\PYZcb{\char`\}}
\def\PYZca{\char`\^}
\def\PYZam{\char`\&}
\def\PYZlt{\char`\<}
\def\PYZgt{\char`\>}
\def\PYZsh{\char`\#}
\def\PYZpc{\char`\%}
\def\PYZdl{\char`\$}
\def\PYZhy{\char`\-}
\def\PYZsq{\char`\'}
\def\PYZdq{\char`\"}
\def\PYZti{\char`\~}
% for compatibility with earlier versions
\def\PYZat{@}
\def\PYZlb{[}
\def\PYZrb{]}
\makeatother


    % Exact colors from NB
    \definecolor{incolor}{rgb}{0.0, 0.0, 0.5}
    \definecolor{outcolor}{rgb}{0.545, 0.0, 0.0}



    
    % Prevent overflowing lines due to hard-to-break entities
    \sloppy 
    % Setup hyperref package
    \hypersetup{
      breaklinks=true,  % so long urls are correctly broken across lines
      colorlinks=true,
      urlcolor=blue,
      linkcolor=darkorange,
      citecolor=darkgreen,
      }
    % Slightly bigger margins than the latex defaults
    
    \geometry{verbose,tmargin=1in,bmargin=1in,lmargin=1in,rmargin=1in}
    
    

    \begin{document}
    
    
    
    \maketitle
    
    
    \tableofcontents


    
    \section{Problems in the Bicincitta data set from
2013}\label{problems-in-the-bicincitta-data-set-from-2013}

There are problems with the Bicincitta data that we need to address
before loading the data into a reliable and proper data-base. We will
point out these problems using examples, and measure their magnitude
using systematic analysis, and then speculate about the cause behind
these problems.

    \subsection{Data}\label{data}

We load the data from JSONs provided to us by Bicincitta at the end of
April 2015.


    The resulting data is in the form of lists of dictionaries.


    \begin{Verbatim}[commandchars=\\\{\}]
a subnetwork is described by, 
	id
	name

 a station is described by,
	latitude
	name
	id
	longitude
	subnetwork\_id

 a user is described by 
	subnetwork\_id
	gender
	expires
	postal\_code
	address
	id

 a transaction is described by, 
	direction
	user\_id
	event\_time
	created\_at
	updated\_at
	station\_id
	id
    \end{Verbatim}

    The resulting dictionaries have ids that are UTF-8 strings. We change
these to integers to make our work easier. The addersses in the user
data are web-quotes, which we need to \emph{unquote}. We will also
unquote the station names, just to be safe. There are keys in a
transaction that do not seem to correspond to the data, but refer to the
time at which the data was loaded into the JSON provided to us.We will
drop these variables, and change the \emph{event\_time} to a time
object.



    



    \subsection{Who are the users?}\label{who-are-the-users}

The simplest question may be the fraction of females vs males,


    \begin{Verbatim}[commandchars=\\\{\}]
Of all the users  60  percent are female  and  39  percent males.
    \end{Verbatim}

    It would be interesting if 60\% of the users were in fact female.
However, as we will see later there seems to be a problem of user
duplicacy biased towards females.

    \subsection{Subnetworks for stations users, and
transactions}\label{subnetworks-for-stations-users-and-transactions}

We have a data table for subnetworks, which contains the subnetwork's id
and name. Both users and stations have been assigned a
\emph{subnetwork\_id} which should be an integer pointing to the
\emph{id} variable in the subnetwork table. We would expect all the
subnetworks in this table to be conceptually equivalent. Thus the
subnetworks \emph{PubliBike} and \emph{Campus} should refer to the same
concept of a subnetwork. However a peek at the data hints against this
assumption. \textbf{It seems that there are two distinct concepts of a
subnetwork in the subnetwork table.} There are several lines of evidence
leading us to this conclusion.

First of all, only 11 of the 18 subnetworks have a station assigned to
them, and 7 have no stations (see table in the appendix).



    While none of the stations have been assigned the subnetwork, most of
the users are in subnetwork PubliBike,


    \begin{Verbatim}[commandchars=\\\{\}]
Number of users from subnetwork PubliBike is 58927
    \end{Verbatim}

    The subnetworks thus appear to mean two different things:

\begin{enumerate}
\def\labelenumi{\arabic{enumi}.}
\item
  a geographic subnetwork that has stations
\item
  an administrative subnetwork that is assigned to a user when she signs
  up.
\end{enumerate}






    So we assign each transaction an administrative and a geographic
subnetwork. The administrative subnetwork is the subnetwork that the
user of the transaction has been assigned, and the geographic subnetwork
is the subnetwork that the station of the transaction has been assigned.
So administratively, all the transactions are in PubliBike, while
geographically they are in 11 different subnetworks. In the appendix we
present a table for subnetworks, showing the number of transactions that
fall in a subnetwork, both administratively and geographically, along
with the number of users.

Looking at the actual number of users who have registered a transaction
makes us question the validity of the user data base.


    \begin{Verbatim}[commandchars=\\\{\}]
Fraction of users who have registered a transaction 0.10673152586
    \end{Verbatim}

    With only 10\% users with registered transactions, where are the
remaining 90\% users from ? Are they left-overs from previous versions
of the system? Or is there an error in the database? Additionally, all
the users with the transactions have been assigned the subnetwork
\emph{PubliBike}.





            \begin{Verbatim}[commandchars=\\\{\}]
{\color{outcolor}Out[{\color{outcolor}126}]:} <matplotlib.axes.\_subplots.AxesSubplot at 0x11a0ec790>
\end{Verbatim}
        
    \begin{center}
    \adjustimage{max size={0.9\linewidth}{0.9\paperheight}}{bicincittaProblems_files/bicincittaProblems_31_1.png}
    \end{center}
    { \hspace*{\fill} \\}
    
    \begin{center}
    \adjustimage{max size={0.9\linewidth}{0.9\paperheight}}{bicincittaProblems_files/bicincittaProblems_31_2.png}
    \end{center}
    { \hspace*{\fill} \\}
    
    \begin{center}
    \adjustimage{max size={0.9\linewidth}{0.9\paperheight}}{bicincittaProblems_files/bicincittaProblems_31_3.png}
    \end{center}
    { \hspace*{\fill} \\}
    
    \subsection{User addresses}\label{user-addresses}

There are several problems associated with user addresses. We have
already noticed, and fixed, that the provided addresses in the JSON have
not been \emph{unquoted} from their web encoding. Here we continue to
explore other problems that may arise in the addresses.

We want to count the number of users at one address. Because the
addresses have been provided as strings, we have to be able to aggregate
all address strings that describe the same address. We have written a
python function to do this task, which takes the address and postal-code
strings to provide a combined string taking into account some empirical
disambiguation criteria such as \emph{Av, Ave}, for \emph{Avenue}.

Addresses are not available for all the users.


    \begin{Verbatim}[commandchars=\\\{\}]
Number of users with available address 21659  of which only  15446  unique
    \end{Verbatim}


    What fraction of unique addresses have multiple users?


    \begin{Verbatim}[commandchars=\\\{\}]
0.230545124951
    \end{Verbatim}

    How many users at addresses with multiple users?


    \begin{Verbatim}[commandchars=\\\{\}]
9774
    \end{Verbatim}

    which corresponds to a fraction of all users with available address,


    \begin{Verbatim}[commandchars=\\\{\}]
0.451267371531
    \end{Verbatim}

    Multiple users at the same address could be actual multiple people, or
multiple registrations by the same person, or a database error. We can
consider as an example the address with the most multiplicity of 53,


    \begin{Verbatim}[commandchars=\\\{\}]
address:  via lambertenghi 1; 6900 , number of users:  53
    \end{Verbatim}

    We could say more about the multiple users at the same address if we
look at their transactions. However as it turns out, we \textbf{do not
have addresses for users who have registered transactions in the data},




    There are as many as 37 users assigned to the same address that also
have more than subnetwork assigned. Addresses with several users might
represent problems of multiple subscription. For example, if we look at
addresses with more than 10 users,

    we see that the user is over-whelmingly females. However, a look at the
lower end of such addresses seems alright,


            \begin{Verbatim}[commandchars=\\\{\}]
{\color{outcolor}Out[{\color{outcolor}807}]:}                              address  numFemales  numMales  numUsers  \textbackslash{}
          3                 poudrière 24; 1950           3         0         3   
          61          avenue beaulieu 20; 1004           2         1         3   
          23   avenue louis-ruchonnet 31; 1003           2         1         3   
          104               eichenweg 12; 1718           2         1         3   
          67        rue saint-rochemin 5; 1004           2         1         3   
          
                                  subnetworks  
          3     set([Campus, Valais Central])  
          61   set([Lausanne-Morges, Campus])  
          23   set([Lausanne-Morges, Campus])  
          104   set([Agglo Fribourg, Campus])  
          67   set([Lausanne-Morges, Campus])  
\end{Verbatim}
        
    These particular addresses appear sensible. There could be more than one
person living at these addresses who have signed up with the bike
system, albeit in different subnetworks. Or may be it is the same person
with 2 different sign-ups in two different sub-networks. This raises the
question: \textbf{How are users registered by the system? One individual
= one signup? Or does a user need a sign-up for each subnetwork that she
wants to use?} If it is the latter, then the provided \emph{user\_ids}
become less useful, because the same individual will appear as different
users according to the \emph{user\_ids}.


            \begin{Verbatim}[commandchars=\\\{\}]
{\color{outcolor}Out[{\color{outcolor}808}]:}                           address  numFemales  numMales  numUsers
          18       via lambertenghi 1; 6900          52         1        53
          2288  chemin des falaises 3; 1005          52         0        52
          1349   chemin des berges 12; 1022          41         0        41
          2150     avenue des bains 9; 1007          37         0        37
          332      via monte carmen 4; 6900          33         0        33
          287      route cantonale 33; 1025          25         0        25
          1444      via madonnetta 23; 6900          24         0        24
          1649     place du tunnel 17; 1005          23         0        23
          1826    avenue des bains 11; 1007          23         0        23
          1997       rue de genève 76; 1004          22         0        22
          1801     route cantonale 35; 1025          22         0        22
          2534           via zurigo 1; 6900          20         1        21
\end{Verbatim}
        

            \begin{Verbatim}[commandchars=\\\{\}]
{\color{outcolor}Out[{\color{outcolor}809}]:}                         address  numFemales  numMales  numUsers
          0      bonne-espérance 28; 1006           1         0         1
          1     37 route cantonnale; 1025           1         0         1
          2     avenue de la dôle 4; 1005           1         0         1
          3             abbesses 21; 2012           1         0         1
          4  chemin de ponfilet 100; 1093           0         1         1
\end{Verbatim}
        
    \subsection{Network Usage}\label{network-usage}

In addition to how many transactions, users for each subnetwork, we can
look at the number of stations and geographic subnetwork that a user
uses.


            \begin{Verbatim}[commandchars=\\\{\}]
{\color{outcolor}Out[{\color{outcolor}77}]:} <matplotlib.axes.\_subplots.AxesSubplot at 0x113969410>
\end{Verbatim}
        
    \begin{center}
    \adjustimage{max size={0.9\linewidth}{0.9\paperheight}}{bicincittaProblems_files/bicincittaProblems_54_1.png}
    \end{center}
    { \hspace*{\fill} \\}
    

            \begin{Verbatim}[commandchars=\\\{\}]
{\color{outcolor}Out[{\color{outcolor}78}]:} <matplotlib.axes.\_subplots.AxesSubplot at 0x115d75750>
\end{Verbatim}
        
    \begin{center}
    \adjustimage{max size={0.9\linewidth}{0.9\paperheight}}{bicincittaProblems_files/bicincittaProblems_55_1.png}
    \end{center}
    { \hspace*{\fill} \\}
    
    \section{Appendix}\label{appendix}


    \begin{Verbatim}[commandchars=\\\{\}]
Table characterizing subnetworks
    \end{Verbatim}

            \begin{Verbatim}[commandchars=\\\{\}]
{\color{outcolor}Out[{\color{outcolor}76}]:}                      id               name  numStations  numTrxns\_admin  \textbackslash{}
         name                                                                      
         La Cote               1            La Cote           13               0   
         Bulle                 3              Bulle            2               0   
         Les Lacs-Romont       4    Les Lacs-Romont            9               0   
         Bâle                  5               Bâle            0               0   
         Riviera              11            Riviera            5               0   
         Morges               15             Morges            0               0   
         Chablais              6           Chablais           10               0   
         Ouchy                16              Ouchy            0               0   
         Paradiso             17           Paradiso            0               0   
         Vevey                14              Vevey            0               0   
         unknown            2011            unknown            0               0   
         unknown            2005            unknown            0               0   
         Valais Central        7     Valais Central            7               0   
         Cern                 18               Cern            0               0   
         Yverdon-les-Bains     8  Yverdon-les-Bains            9               0   
         Agglo Fribourg        2     Agglo Fribourg           10               0   
         Lausanne-Morges       9    Lausanne-Morges           11               0   
         Lugano-Paradiso      12    Lugano-Paradiso           13               0   
         Campus               10             Campus           15               0   
         PubliBike            13          PubliBike            0          291134   
         
                            numTrxns\_geo  numUsers  
         name                                       
         La Cote                   71292         0  
         Bulle                       566         0  
         Les Lacs-Romont            4964         0  
         Bâle                          0         0  
         Riviera                   11576         0  
         Morges                        0         0  
         Chablais                   2377         3  
         Ouchy                         0         3  
         Paradiso                      0         3  
         Vevey                         0         4  
         unknown                       0       152  
         unknown                       0       469  
         Valais Central             5077       530  
         Cern                          0       721  
         Yverdon-les-Bains         33227       773  
         Agglo Fribourg            17630       810  
         Lausanne-Morges           12157      2183  
         Lugano-Paradiso           66415      3425  
         Campus                    65853     15871  
         PubliBike                     0     58927  
\end{Verbatim}
        


    % Add a bibliography block to the postdoc
    
    
    
    \end{document}
