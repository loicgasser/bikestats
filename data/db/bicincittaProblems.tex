
% Default to the notebook output style

    


% Inherit from the specified cell style.




    
\documentclass{report}

    
    
    \usepackage{graphicx} % Used to insert images
    \usepackage{adjustbox} % Used to constrain images to a maximum size 
    \usepackage{color} % Allow colors to be defined
    \usepackage{enumerate} % Needed for markdown enumerations to work
    \usepackage{geometry} % Used to adjust the document margins
    \usepackage{amsmath} % Equations
    \usepackage{amssymb} % Equations
    \usepackage{eurosym} % defines \euro
    \usepackage[mathletters]{ucs} % Extended unicode (utf-8) support
    \usepackage[utf8x]{inputenc} % Allow utf-8 characters in the tex document
    \usepackage{fancyvrb} % verbatim replacement that allows latex
    \usepackage{grffile} % extends the file name processing of package graphics 
                         % to support a larger range 
    % The hyperref package gives us a pdf with properly built
    % internal navigation ('pdf bookmarks' for the table of contents,
    % internal cross-reference links, web links for URLs, etc.)
    \usepackage{hyperref}
    \usepackage{longtable} % longtable support required by pandoc >1.10
    \usepackage{booktabs}  % table support for pandoc > 1.12.2
    

    
    
    \definecolor{orange}{cmyk}{0,0.4,0.8,0.2}
    \definecolor{darkorange}{rgb}{.71,0.21,0.01}
    \definecolor{darkgreen}{rgb}{.12,.54,.11}
    \definecolor{myteal}{rgb}{.26, .44, .56}
    \definecolor{gray}{gray}{0.45}
    \definecolor{lightgray}{gray}{.95}
    \definecolor{mediumgray}{gray}{.8}
    \definecolor{inputbackground}{rgb}{.95, .95, .85}
    \definecolor{outputbackground}{rgb}{.95, .95, .95}
    \definecolor{traceback}{rgb}{1, .95, .95}
    % ansi colors
    \definecolor{red}{rgb}{.6,0,0}
    \definecolor{green}{rgb}{0,.65,0}
    \definecolor{brown}{rgb}{0.6,0.6,0}
    \definecolor{blue}{rgb}{0,.145,.698}
    \definecolor{purple}{rgb}{.698,.145,.698}
    \definecolor{cyan}{rgb}{0,.698,.698}
    \definecolor{lightgray}{gray}{0.5}
    
    % bright ansi colors
    \definecolor{darkgray}{gray}{0.25}
    \definecolor{lightred}{rgb}{1.0,0.39,0.28}
    \definecolor{lightgreen}{rgb}{0.48,0.99,0.0}
    \definecolor{lightblue}{rgb}{0.53,0.81,0.92}
    \definecolor{lightpurple}{rgb}{0.87,0.63,0.87}
    \definecolor{lightcyan}{rgb}{0.5,1.0,0.83}
    
    % commands and environments needed by pandoc snippets
    % extracted from the output of `pandoc -s`
    \DefineVerbatimEnvironment{Highlighting}{Verbatim}{commandchars=\\\{\}}
    % Add ',fontsize=\small' for more characters per line
    \newenvironment{Shaded}{}{}
    \newcommand{\KeywordTok}[1]{\textcolor[rgb]{0.00,0.44,0.13}{\textbf{{#1}}}}
    \newcommand{\DataTypeTok}[1]{\textcolor[rgb]{0.56,0.13,0.00}{{#1}}}
    \newcommand{\DecValTok}[1]{\textcolor[rgb]{0.25,0.63,0.44}{{#1}}}
    \newcommand{\BaseNTok}[1]{\textcolor[rgb]{0.25,0.63,0.44}{{#1}}}
    \newcommand{\FloatTok}[1]{\textcolor[rgb]{0.25,0.63,0.44}{{#1}}}
    \newcommand{\CharTok}[1]{\textcolor[rgb]{0.25,0.44,0.63}{{#1}}}
    \newcommand{\StringTok}[1]{\textcolor[rgb]{0.25,0.44,0.63}{{#1}}}
    \newcommand{\CommentTok}[1]{\textcolor[rgb]{0.38,0.63,0.69}{\textit{{#1}}}}
    \newcommand{\OtherTok}[1]{\textcolor[rgb]{0.00,0.44,0.13}{{#1}}}
    \newcommand{\AlertTok}[1]{\textcolor[rgb]{1.00,0.00,0.00}{\textbf{{#1}}}}
    \newcommand{\FunctionTok}[1]{\textcolor[rgb]{0.02,0.16,0.49}{{#1}}}
    \newcommand{\RegionMarkerTok}[1]{{#1}}
    \newcommand{\ErrorTok}[1]{\textcolor[rgb]{1.00,0.00,0.00}{\textbf{{#1}}}}
    \newcommand{\NormalTok}[1]{{#1}}
    
    % Define a nice break command that doesn't care if a line doesn't already
    % exist.
    \def\br{\hspace*{\fill} \\* }
    % Math Jax compatability definitions
    \def\gt{>}
    \def\lt{<}
    % Document parameters
    \title{bicincittaProblems}
    
    
    

    % Pygments definitions
    
\makeatletter
\def\PY@reset{\let\PY@it=\relax \let\PY@bf=\relax%
    \let\PY@ul=\relax \let\PY@tc=\relax%
    \let\PY@bc=\relax \let\PY@ff=\relax}
\def\PY@tok#1{\csname PY@tok@#1\endcsname}
\def\PY@toks#1+{\ifx\relax#1\empty\else%
    \PY@tok{#1}\expandafter\PY@toks\fi}
\def\PY@do#1{\PY@bc{\PY@tc{\PY@ul{%
    \PY@it{\PY@bf{\PY@ff{#1}}}}}}}
\def\PY#1#2{\PY@reset\PY@toks#1+\relax+\PY@do{#2}}

\expandafter\def\csname PY@tok@gd\endcsname{\def\PY@tc##1{\textcolor[rgb]{0.63,0.00,0.00}{##1}}}
\expandafter\def\csname PY@tok@gu\endcsname{\let\PY@bf=\textbf\def\PY@tc##1{\textcolor[rgb]{0.50,0.00,0.50}{##1}}}
\expandafter\def\csname PY@tok@gt\endcsname{\def\PY@tc##1{\textcolor[rgb]{0.00,0.27,0.87}{##1}}}
\expandafter\def\csname PY@tok@gs\endcsname{\let\PY@bf=\textbf}
\expandafter\def\csname PY@tok@gr\endcsname{\def\PY@tc##1{\textcolor[rgb]{1.00,0.00,0.00}{##1}}}
\expandafter\def\csname PY@tok@cm\endcsname{\let\PY@it=\textit\def\PY@tc##1{\textcolor[rgb]{0.25,0.50,0.50}{##1}}}
\expandafter\def\csname PY@tok@vg\endcsname{\def\PY@tc##1{\textcolor[rgb]{0.10,0.09,0.49}{##1}}}
\expandafter\def\csname PY@tok@m\endcsname{\def\PY@tc##1{\textcolor[rgb]{0.40,0.40,0.40}{##1}}}
\expandafter\def\csname PY@tok@mh\endcsname{\def\PY@tc##1{\textcolor[rgb]{0.40,0.40,0.40}{##1}}}
\expandafter\def\csname PY@tok@go\endcsname{\def\PY@tc##1{\textcolor[rgb]{0.53,0.53,0.53}{##1}}}
\expandafter\def\csname PY@tok@ge\endcsname{\let\PY@it=\textit}
\expandafter\def\csname PY@tok@vc\endcsname{\def\PY@tc##1{\textcolor[rgb]{0.10,0.09,0.49}{##1}}}
\expandafter\def\csname PY@tok@il\endcsname{\def\PY@tc##1{\textcolor[rgb]{0.40,0.40,0.40}{##1}}}
\expandafter\def\csname PY@tok@cs\endcsname{\let\PY@it=\textit\def\PY@tc##1{\textcolor[rgb]{0.25,0.50,0.50}{##1}}}
\expandafter\def\csname PY@tok@cp\endcsname{\def\PY@tc##1{\textcolor[rgb]{0.74,0.48,0.00}{##1}}}
\expandafter\def\csname PY@tok@gi\endcsname{\def\PY@tc##1{\textcolor[rgb]{0.00,0.63,0.00}{##1}}}
\expandafter\def\csname PY@tok@gh\endcsname{\let\PY@bf=\textbf\def\PY@tc##1{\textcolor[rgb]{0.00,0.00,0.50}{##1}}}
\expandafter\def\csname PY@tok@ni\endcsname{\let\PY@bf=\textbf\def\PY@tc##1{\textcolor[rgb]{0.60,0.60,0.60}{##1}}}
\expandafter\def\csname PY@tok@nl\endcsname{\def\PY@tc##1{\textcolor[rgb]{0.63,0.63,0.00}{##1}}}
\expandafter\def\csname PY@tok@nn\endcsname{\let\PY@bf=\textbf\def\PY@tc##1{\textcolor[rgb]{0.00,0.00,1.00}{##1}}}
\expandafter\def\csname PY@tok@no\endcsname{\def\PY@tc##1{\textcolor[rgb]{0.53,0.00,0.00}{##1}}}
\expandafter\def\csname PY@tok@na\endcsname{\def\PY@tc##1{\textcolor[rgb]{0.49,0.56,0.16}{##1}}}
\expandafter\def\csname PY@tok@nb\endcsname{\def\PY@tc##1{\textcolor[rgb]{0.00,0.50,0.00}{##1}}}
\expandafter\def\csname PY@tok@nc\endcsname{\let\PY@bf=\textbf\def\PY@tc##1{\textcolor[rgb]{0.00,0.00,1.00}{##1}}}
\expandafter\def\csname PY@tok@nd\endcsname{\def\PY@tc##1{\textcolor[rgb]{0.67,0.13,1.00}{##1}}}
\expandafter\def\csname PY@tok@ne\endcsname{\let\PY@bf=\textbf\def\PY@tc##1{\textcolor[rgb]{0.82,0.25,0.23}{##1}}}
\expandafter\def\csname PY@tok@nf\endcsname{\def\PY@tc##1{\textcolor[rgb]{0.00,0.00,1.00}{##1}}}
\expandafter\def\csname PY@tok@si\endcsname{\let\PY@bf=\textbf\def\PY@tc##1{\textcolor[rgb]{0.73,0.40,0.53}{##1}}}
\expandafter\def\csname PY@tok@s2\endcsname{\def\PY@tc##1{\textcolor[rgb]{0.73,0.13,0.13}{##1}}}
\expandafter\def\csname PY@tok@vi\endcsname{\def\PY@tc##1{\textcolor[rgb]{0.10,0.09,0.49}{##1}}}
\expandafter\def\csname PY@tok@nt\endcsname{\let\PY@bf=\textbf\def\PY@tc##1{\textcolor[rgb]{0.00,0.50,0.00}{##1}}}
\expandafter\def\csname PY@tok@nv\endcsname{\def\PY@tc##1{\textcolor[rgb]{0.10,0.09,0.49}{##1}}}
\expandafter\def\csname PY@tok@s1\endcsname{\def\PY@tc##1{\textcolor[rgb]{0.73,0.13,0.13}{##1}}}
\expandafter\def\csname PY@tok@kd\endcsname{\let\PY@bf=\textbf\def\PY@tc##1{\textcolor[rgb]{0.00,0.50,0.00}{##1}}}
\expandafter\def\csname PY@tok@sh\endcsname{\def\PY@tc##1{\textcolor[rgb]{0.73,0.13,0.13}{##1}}}
\expandafter\def\csname PY@tok@sc\endcsname{\def\PY@tc##1{\textcolor[rgb]{0.73,0.13,0.13}{##1}}}
\expandafter\def\csname PY@tok@sx\endcsname{\def\PY@tc##1{\textcolor[rgb]{0.00,0.50,0.00}{##1}}}
\expandafter\def\csname PY@tok@bp\endcsname{\def\PY@tc##1{\textcolor[rgb]{0.00,0.50,0.00}{##1}}}
\expandafter\def\csname PY@tok@c1\endcsname{\let\PY@it=\textit\def\PY@tc##1{\textcolor[rgb]{0.25,0.50,0.50}{##1}}}
\expandafter\def\csname PY@tok@kc\endcsname{\let\PY@bf=\textbf\def\PY@tc##1{\textcolor[rgb]{0.00,0.50,0.00}{##1}}}
\expandafter\def\csname PY@tok@c\endcsname{\let\PY@it=\textit\def\PY@tc##1{\textcolor[rgb]{0.25,0.50,0.50}{##1}}}
\expandafter\def\csname PY@tok@mf\endcsname{\def\PY@tc##1{\textcolor[rgb]{0.40,0.40,0.40}{##1}}}
\expandafter\def\csname PY@tok@err\endcsname{\def\PY@bc##1{\setlength{\fboxsep}{0pt}\fcolorbox[rgb]{1.00,0.00,0.00}{1,1,1}{\strut ##1}}}
\expandafter\def\csname PY@tok@mb\endcsname{\def\PY@tc##1{\textcolor[rgb]{0.40,0.40,0.40}{##1}}}
\expandafter\def\csname PY@tok@ss\endcsname{\def\PY@tc##1{\textcolor[rgb]{0.10,0.09,0.49}{##1}}}
\expandafter\def\csname PY@tok@sr\endcsname{\def\PY@tc##1{\textcolor[rgb]{0.73,0.40,0.53}{##1}}}
\expandafter\def\csname PY@tok@mo\endcsname{\def\PY@tc##1{\textcolor[rgb]{0.40,0.40,0.40}{##1}}}
\expandafter\def\csname PY@tok@kn\endcsname{\let\PY@bf=\textbf\def\PY@tc##1{\textcolor[rgb]{0.00,0.50,0.00}{##1}}}
\expandafter\def\csname PY@tok@mi\endcsname{\def\PY@tc##1{\textcolor[rgb]{0.40,0.40,0.40}{##1}}}
\expandafter\def\csname PY@tok@gp\endcsname{\let\PY@bf=\textbf\def\PY@tc##1{\textcolor[rgb]{0.00,0.00,0.50}{##1}}}
\expandafter\def\csname PY@tok@o\endcsname{\def\PY@tc##1{\textcolor[rgb]{0.40,0.40,0.40}{##1}}}
\expandafter\def\csname PY@tok@kr\endcsname{\let\PY@bf=\textbf\def\PY@tc##1{\textcolor[rgb]{0.00,0.50,0.00}{##1}}}
\expandafter\def\csname PY@tok@s\endcsname{\def\PY@tc##1{\textcolor[rgb]{0.73,0.13,0.13}{##1}}}
\expandafter\def\csname PY@tok@kp\endcsname{\def\PY@tc##1{\textcolor[rgb]{0.00,0.50,0.00}{##1}}}
\expandafter\def\csname PY@tok@w\endcsname{\def\PY@tc##1{\textcolor[rgb]{0.73,0.73,0.73}{##1}}}
\expandafter\def\csname PY@tok@kt\endcsname{\def\PY@tc##1{\textcolor[rgb]{0.69,0.00,0.25}{##1}}}
\expandafter\def\csname PY@tok@ow\endcsname{\let\PY@bf=\textbf\def\PY@tc##1{\textcolor[rgb]{0.67,0.13,1.00}{##1}}}
\expandafter\def\csname PY@tok@sb\endcsname{\def\PY@tc##1{\textcolor[rgb]{0.73,0.13,0.13}{##1}}}
\expandafter\def\csname PY@tok@k\endcsname{\let\PY@bf=\textbf\def\PY@tc##1{\textcolor[rgb]{0.00,0.50,0.00}{##1}}}
\expandafter\def\csname PY@tok@se\endcsname{\let\PY@bf=\textbf\def\PY@tc##1{\textcolor[rgb]{0.73,0.40,0.13}{##1}}}
\expandafter\def\csname PY@tok@sd\endcsname{\let\PY@it=\textit\def\PY@tc##1{\textcolor[rgb]{0.73,0.13,0.13}{##1}}}

\def\PYZbs{\char`\\}
\def\PYZus{\char`\_}
\def\PYZob{\char`\{}
\def\PYZcb{\char`\}}
\def\PYZca{\char`\^}
\def\PYZam{\char`\&}
\def\PYZlt{\char`\<}
\def\PYZgt{\char`\>}
\def\PYZsh{\char`\#}
\def\PYZpc{\char`\%}
\def\PYZdl{\char`\$}
\def\PYZhy{\char`\-}
\def\PYZsq{\char`\'}
\def\PYZdq{\char`\"}
\def\PYZti{\char`\~}
% for compatibility with earlier versions
\def\PYZat{@}
\def\PYZlb{[}
\def\PYZrb{]}
\makeatother


    % Exact colors from NB
    \definecolor{incolor}{rgb}{0.0, 0.0, 0.5}
    \definecolor{outcolor}{rgb}{0.545, 0.0, 0.0}



    
    % Prevent overflowing lines due to hard-to-break entities
    \sloppy 
    % Setup hyperref package
    \hypersetup{
      breaklinks=true,  % so long urls are correctly broken across lines
      colorlinks=true,
      urlcolor=blue,
      linkcolor=darkorange,
      citecolor=darkgreen,
      }
    % Slightly bigger margins than the latex defaults
    
    \geometry{verbose,tmargin=1in,bmargin=1in,lmargin=1in,rmargin=1in}
    
    

    \begin{document}
    
    
    
    \maketitle
    
    
    \tableofcontents


    
    \section{Problems in the Bicincitta data set from
2013}\label{problems-in-the-bicincitta-data-set-from-2013}

There are problems with the Bicincitta data that we need to address
before loading the data into a reliable and proper data-base. We will
point out these problems using examples, and measure their magnitude
using systematic analysis, and then speculate about the source of these
problems.

    \subsection{Data}\label{data}

We will load the data from JSONs provided to us by Bicincitta at the end
of April 2015.


    The resulting data is in the form of lists of dictionaries. We can take
a peek at the keys in each of the four data types, by creating a data
frame and displaying the first few rows.


    \begin{Verbatim}[commandchars=\\\{\}]
a subnetwork is described by, 
	id
	name

 a station is described by,
	name
	longitude
	subnetwork\_name
	latitude
	id
	subnetwork\_id

 a user is described by 
	subnetwork\_id
	gender
	expires
	postal\_code
	subnetwork\_name
	address
	id

 a transaction is described by, 
	direction
	user\_id
	station\_id
	event\_time
	id
    \end{Verbatim}

    The resulting dictionaries have ids that are UTF-8 strings. We can
change these to integers to make our work easier,


    There are keys in a transaction that do not seem to correspond to the
data, but refer to the time at which the data was loaded into the JSON
provided to us. We will drop these variables, and change the
\emph{event\_time} to a time object. Following that we will sort the
transactions by the event time


    We also sort the subnetworks and stations by their \emph{id}.


    Stations and users have been assigned a \emph{subnetwork\_id} in the
data. We can add the subnetwork name to these data,


    \subsection{Who are the users?}\label{who-are-the-users}

The simplest question may be the fraction of females vs males,


    \begin{Verbatim}[commandchars=\\\{\}]
Of all the users  60  percent are female  and  39  percent males.
    \end{Verbatim}

    It would be interesting if 60\% of the users were in fact female.
However, as we will see later there seems to be a problem of user
duplicacy biased towards females.

    \subsection{Subnetworks for stations and
users}\label{subnetworks-for-stations-and-users}

Are the \emph{subnetwork\_id}s for stations and users sensible? The
subnetworks that the stations fall in are


    \begin{Verbatim}[commandchars=\\\{\}]
Subnetworks that have stations assigned to them
    \end{Verbatim}

            \begin{Verbatim}[commandchars=\\\{\}]
	{} {}     id               name
          0    1            La Cote
          1    2     Agglo Fribourg
          2    3              Bulle
          3    4    Les Lacs-Romont
          4    6           Chablais
          5    7     Valais Central
          6    8  Yverdon-les-Bains
          7    9    Lausanne-Morges
          8   10             Campus
          9   11            Riviera
          10  12    Lugano-Paradiso
\end{Verbatim}
        
    We see that stations cover only 11 of the 18 subnetworks. Looking at the
subnetworks with no stations,


    \begin{Verbatim}[commandchars=\\\{\}]
Subnetworks without any assigned stations
    \end{Verbatim}

            \begin{Verbatim}[commandchars=\\\{\}]
	     id       name
          0   5       Bâle
          1  13  PubliBike
          2  14      Vevey
          3  15     Morges
          4  16      Ouchy
          5  17   Paradiso
          6  18       Cern
\end{Verbatim}
        
    we can see why there are no stations corresponding to these subnetworks.
Basel, and Cern because Bicincitta have not given us data for these
regions. Vevey, Morges, Ouchy, and Paradiso include stations subsumed by
other subnetworks. These networks may be a remnant from previous
versions of the data. This leaves \textbf{PubliBike} unexplained. As it
turns out, there are users that have been assigned the subnetwork
PubliBike (\emph{id} 13). In fact we see later that the users who have
registered transactions in the data have been assigned only PubliBike,
and no other subnetwork.


    \begin{Verbatim}[commandchars=\\\{\}]
Number of users from subnetwork PubliBike 58927
    \end{Verbatim}

    Users that have been assigned subnetwork PubliBike compose 70\% of the
total users in the data. However we are not going to see PubliBike in
any of their transactions as there are no stations for the subnetwork
PubliBike! What about other subnetworks without stations?


    \begin{Verbatim}[commandchars=\\\{\}]
subnetworks assigned to users
    \end{Verbatim}

            \begin{Verbatim}[commandchars=\\\{\}]
{\color{outcolor}Out[{\color{outcolor}607}]:}     id               name
          0    2     Agglo Fribourg
          1    6           Chablais
          2    7     Valais Central
          3    8  Yverdon-les-Bains
          4    9    Lausanne-Morges
          5   10             Campus
          6   12    Lugano-Paradiso
          7   13          PubliBike
          8   14              Vevey
          9   16              Ouchy
          10  17           Paradiso
          11  18               Cern
\end{Verbatim}
        

    \begin{Verbatim}[commandchars=\\\{\}]
subnets without any users
    \end{Verbatim}

            \begin{Verbatim}[commandchars=\\\{\}]
{\color{outcolor}Out[{\color{outcolor}608}]:}    id             name
          0   1          La Cote
          1   3            Bulle
          2   4  Les Lacs-Romont
          3   5             Bâle
          4  11          Riviera
          5  15           Morges
\end{Verbatim}
        
    Comparing the subnets for users to subnets with stations we see that
there are \textbf{only 7 subnets} for which we have stations as well as
users,


    \begin{Verbatim}[commandchars=\\\{\}]
subnets with stations as well as users
    \end{Verbatim}

            \begin{Verbatim}[commandchars=\\\{\}]
{\color{outcolor}Out[{\color{outcolor}609}]:}    id               name
          0   2     Agglo Fribourg
          1   6           Chablais
          2   7     Valais Central
          3   8  Yverdon-les-Bains
          4   9    Lausanne-Morges
          5  10             Campus
          6  12    Lugano-Paradiso
\end{Verbatim}
        
    As a summary, let us tabulate the fraction of users in each of the
subnets,



            \begin{Verbatim}[commandchars=\\\{\}]
{\color{outcolor}Out[{\color{outcolor}611}]:}       id               name  numUsers
          0      1            La Cote         0
          1      3              Bulle         0
          2      4    Les Lacs-Romont         0
          3      5               Bâle         0
          4     11            Riviera         0
          5     15             Morges         0
          6      6           Chablais         3
          7     16              Ouchy         3
          8     17           Paradiso         3
          9     14              Vevey         4
          10  2011            unknown       152
          11  2005            unknown       469
          12     7     Valais Central       530
          13    18               Cern       721
          14     8  Yverdon-les-Bains       773
          15     2     Agglo Fribourg       810
          16     9    Lausanne-Morges      2183
          17    12    Lugano-Paradiso      3425
          18    10             Campus     15871
          19    13          PubliBike     58927
\end{Verbatim}
        

        
    \begin{center}
    \adjustimage{max size={0.9\linewidth}{0.9\paperheight}}{bicincittaProblems_files/bicincittaProblems_30_1.png}
    \end{center}
    { \hspace*{\fill} \\}
    
    So, most of the users are in PubliBike, which could create a problem as
there are no stations associated to PubliBike. Lets first look at the
transactions before we try to find a solution to this problem.

\subsection{Transaction users, stations, and
subnetworks}\label{transaction-users-stations-and-subnetworks}

We begin by looking at how many users actually use the bike system (
have valid transactions)


    \begin{Verbatim}[commandchars=\\\{\}]
fraction of users who have registered a transaction 0.10673152586
    \end{Verbatim}

    Only 10\% users have registered a transaction! Are all the transactions
of a user in the same subnetwork?



            \begin{Verbatim}[commandchars=\\\{\}]
{\color{outcolor}Out[{\color{outcolor}800}]:}       user\_id  assigned\_subnet  numStations  numSubnets  numTrxns
          3462   108132               13           56           8      2180
          4469   111523               13           46           7      2110
          5399    84545               13           43           7      5134
          3463   108133               13           43           6      3280
          4673   112287               13           23           3       540
\end{Verbatim}
        

        
    \begin{center}
    \adjustimage{max size={0.9\linewidth}{0.9\paperheight}}{bicincittaProblems_files/bicincittaProblems_36_1.png}
    \end{center}
    { \hspace*{\fill} \\}
    

    \adjustimage{max size={0.9\linewidth}{0.9\paperheight}}{bicincittaProblems_files/bicincittaProblems_37_1.png}
    \end{center}
    { \hspace*{\fill} \\}
    
    A simple question might guide us. How many transactions belong to
Publibike users ( who have subnetwork\_id PubliBike) ?


    \begin{Verbatim}[commandchars=\\\{\}]
The subnetworks of the users who make transactions make a set([13])
    \end{Verbatim}

    So, \textbf{from the view point of the users, all the transactions are
in the subnetwork PubliBike}

    How many transactions in a subnetwork of the stations?


    Now we can make a table for subnetworks, counting the number of trxns
through user and station \emph{subnetwork\_id}, in addition to the
number of users.



        
    \begin{center}
    \adjustimage{max size={0.9\linewidth}{0.9\paperheight}}{bicincittaProblems_files/bicincittaProblems_45_1.png}
    \end{center}
    { \hspace*{\fill} \\}
    
    \begin{center}
    \adjustimage{max size={0.9\linewidth}{0.9\paperheight}}{bicincittaProblems_files/bicincittaProblems_45_2.png}
    \end{center}
    { \hspace*{\fill} \\}
    
    \begin{center}
    \adjustimage{max size={0.9\linewidth}{0.9\paperheight}}{bicincittaProblems_files/bicincittaProblems_45_3.png}
    \end{center}
    { \hspace*{\fill} \\}
    

            \begin{Verbatim}[commandchars=\\\{\}]
{\color{outcolor}Out[{\color{outcolor}764}]:}                      id               name  numStations  numTrxns\_stns  \textbackslash{}
          name                                                                     
          Vevey                14              Vevey            0              0   
          Cern                 18               Cern            0              0   
          unknown            2005            unknown            0              0   
          unknown            2011            unknown            0              0   
          Paradiso             17           Paradiso            0              0   
          Ouchy                16              Ouchy            0              0   
          PubliBike            13          PubliBike            0              0   
          Morges               15             Morges            0              0   
          Bâle                  5               Bâle            0              0   
          Bulle                 3              Bulle            2            566   
          Riviera              11            Riviera            5          11576   
          Valais Central        7     Valais Central            7           5077   
          Les Lacs-Romont       4    Les Lacs-Romont            9           4964   
          Yverdon-les-Bains     8  Yverdon-les-Bains            9          33227   
          Agglo Fribourg        2     Agglo Fribourg           10          17630   
          Chablais              6           Chablais           10           2377   
          Lausanne-Morges       9    Lausanne-Morges           11          12157   
          Lugano-Paradiso      12    Lugano-Paradiso           13          66415   
          La Cote               1            La Cote           13          71292   
          Campus               10             Campus           15          65853   
          
                             numTrxns\_users  numUsers  
          name                                         
          Vevey                           0         4  
          Cern                            0       721  
          unknown                         0       469  
          unknown                         0       152  
          Paradiso                        0         3  
          Ouchy                           0         3  
          PubliBike                  291134     58927  
          Morges                          0         0  
          Bâle                            0         0  
          Bulle                           0         0  
          Riviera                         0         0  
          Valais Central                  0       530  
          Les Lacs-Romont                 0         0  
          Yverdon-les-Bains               0       773  
          Agglo Fribourg                  0       810  
          Chablais                        0         3  
          Lausanne-Morges                 0      2183  
          Lugano-Paradiso                 0      3425  
          La Cote                         0         0  
          Campus                          0     15871  
\end{Verbatim}
        
    \subsection{User addresses}\label{user-addresses}

There are several problems associated with user addresses. We have
already noticed, and fixed, that the provided addresses in the JSON have
not been \emph{unquoted} from their web encoding. Here we continue to
explore other problems that may arise in the addresses.

We want to count the number of users at one address. Because the
addresses have been provided as strings, we have to be able to aggregate
all address strings that describe the same address. We have written a
python function to do this task, which takes the address and postal-code
strings to provide a combined string taking into account some empirical
disambiguation criteria such as \emph{Av, Ave}, for \emph{Avenue}.



    \begin{Verbatim}[commandchars=\\\{\}]
number of users with available address 21659
number of these addresses that are unique 15446
    \end{Verbatim}


    What fraction of unique addresses have multiple users?


    \begin{Verbatim}[commandchars=\\\{\}]
0.230545124951
    \end{Verbatim}

    How many users at addresses with multiple users?


    \begin{Verbatim}[commandchars=\\\{\}]
9774
    \end{Verbatim}

    which corresponds to a fraction of all users with available address,


    \begin{Verbatim}[commandchars=\\\{\}]
0.451267371531
    \end{Verbatim}

    Multiple users at the same address could be actual multiple people, or
multiple registrations by the same person, or a glitch in the data. We
can consider as an example the address with the most multiplicity of 53,


    \begin{Verbatim}[commandchars=\\\{\}]
address:  via lambertenghi 1; 6900 , number of users:  53
    \end{Verbatim}

    We could say more about the multiple users at the same address if we
look at their transactions. However as it turns out, we \textbf{do not
have addresses for users who have registered transactions in the data},


    \begin{Verbatim}[commandchars=\\\{\}]
False
    \end{Verbatim}

    We can look at the subnetwork with addresses assigned to the
\emph{multi} users,


    \begin{Verbatim}[commandchars=\\\{\}]
subnetworks with addresses assigned to multiple users set([2, 7, 8, 9, 10, 12, 18, 2005, 2011])
    \end{Verbatim}

    Some of the multi-user addresses have more than one subnetworks (
through the users at that address)


    \begin{Verbatim}[commandchars=\\\{\}]
Subnetworks for users living at addresses with multiple registered users
    \end{Verbatim}

    There are as many as 37 users assigned to the same address that also
have more than subnetwork assigned. Addresses with several users might
represent problems of multiple subscription. For example, if we look at
addresses with more than 10 users,


            \begin{Verbatim}[commandchars=\\\{\}]
{\color{outcolor}Out[{\color{outcolor}806}]:}                        address  numFemales  numMales  numUsers  \textbackslash{}
          143   avenue des bains 9; 1007          37         0        37   
          40    route cantonale 33; 1025          25         0        25   
          105  avenue des bains 11; 1007          23         0        23   
          48    place du tunnel 17; 1005          23         0        23   
          
                                  subnetworks  
          143  set([Lausanne-Morges, Campus])  
          40   set([Lausanne-Morges, Campus])  
          105  set([Lausanne-Morges, Campus])  
          48   set([Lausanne-Morges, Campus])  
\end{Verbatim}
        
    we see that the user is over-whelmingly females. However, a look at the
lower end of such addresses seems alright,


            \begin{Verbatim}[commandchars=\\\{\}]
{\color{outcolor}Out[{\color{outcolor}807}]:}                              address  numFemales  numMales  numUsers  \textbackslash{}
          3                 poudrière 24; 1950           3         0         3   
          61          avenue beaulieu 20; 1004           2         1         3   
          23   avenue louis-ruchonnet 31; 1003           2         1         3   
          104               eichenweg 12; 1718           2         1         3   
          67        rue saint-rochemin 5; 1004           2         1         3   
          
                                  subnetworks  
          3     set([Campus, Valais Central])  
          61   set([Lausanne-Morges, Campus])  
          23   set([Lausanne-Morges, Campus])  
          104   set([Agglo Fribourg, Campus])  
          67   set([Lausanne-Morges, Campus])  
\end{Verbatim}
        
    These particular addresses appear sensible. There could be more than one
person living at these addresses who have signed up with the bike
system, albeit in different subnetworks. Or may be it is the same person
with 2 different sign-ups in two different sub-networks. This raises the
question: \textbf{How are users registered by the system? One individual
= one signup? Or does a user need a sign-up for each subnetwork that she
wants to use?} If it is the latter, then the provided \emph{user\_ids}
become less useful, because the same individual will appear as different
users according to the \emph{user\_ids}.


            \begin{Verbatim}[commandchars=\\\{\}]
{\color{outcolor}Out[{\color{outcolor}808}]:}                           address  numFemales  numMales  numUsers
          18       via lambertenghi 1; 6900          52         1        53
          2288  chemin des falaises 3; 1005          52         0        52
          1349   chemin des berges 12; 1022          41         0        41
          2150     avenue des bains 9; 1007          37         0        37
          332      via monte carmen 4; 6900          33         0        33
          287      route cantonale 33; 1025          25         0        25
          1444      via madonnetta 23; 6900          24         0        24
          1649     place du tunnel 17; 1005          23         0        23
          1826    avenue des bains 11; 1007          23         0        23
          1997       rue de genève 76; 1004          22         0        22
          1801     route cantonale 35; 1025          22         0        22
          2534           via zurigo 1; 6900          20         1        21
\end{Verbatim}
        

            \begin{Verbatim}[commandchars=\\\{\}]
{\color{outcolor}Out[{\color{outcolor}809}]:}                         address  numFemales  numMales  numUsers
          0      bonne-espérance 28; 1006           1         0         1
          1     37 route cantonnale; 1025           1         0         1
          2     avenue de la dôle 4; 1005           1         0         1
          3             abbesses 21; 2012           1         0         1
          4  chemin de ponfilet 100; 1093           0         1         1
\end{Verbatim}
        

    % Add a bibliography block to the postdoc
    
    
    
    \end{document}
